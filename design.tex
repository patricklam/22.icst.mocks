We next provide some background information so that we can better describe ``classical'' and ``mockist'' views.
Meszaros~\cite{meszaros2007xunit} (and, following him, Fowler~\cite{fowler07:_mocks_arent_stubs}) use the term ``test double'' for the concept of a collaborator object required during testing. When testing a given class \texttt{C}, a developer will need to somehow provide instances of \texttt{C}'s collaborators to exercise \texttt{C}'s functionality. 
One of our key goals is to understand how free developers are to choose between different kinds of test doubles---notably, mocks and stubs/fakes. 

Returning to the original sources, in the first publication on mock objects~\cite{mackinnon00:_endo_testin}, the motivating example is a test case exercising a \texttt{JunitCreatorModel} object; that object takes a workspace object as a parameter to its constructor.

\begin{lstlisting}
  // ...
  JUnitCreatorModel creatorModel = new
        JunitCreatorModel(myMockWorkspace, PACKAGE_NAME);
  try {
    creatorModel.createTestCase
        (EXISTING_CLASS_NAME);
    // ...
\end{lstlisting}

This test needs a workspace object to exercise the model. (In this case, it is not possible to pass \texttt{null} as the constructor parameter, because the \texttt{JunitCreatorModel} calls methods on the workspace.) Stubs, fakes, and mocks are some of the kinds of test doubles. Following Meszaros, a stub's methods return canned answers, while a fake is a simplified but working implementation (e.g. an in-memory database rather than one that stores data to persistent storage). 

(Note that the definition of mock objects has shifted since the initial publication~\cite{mackinnon00:_endo_testin}; their use of the term ``mock'' includes what Meszaros would later call stubs and fakes. In their subsequent work~\cite{freeman04:_mock_roles_objec}, they still refer to stubs, but discourage explicit stub implementations and encourage the recording of mock object behaviours, as seen in their early mock framework jMock as well as modern mock frameworks including Mockito, PowerMock, and EasyMock.)

With the introduction of the concept of mocks came two worldviews mentioned above: the ``classical'' worldview and the ``mockist'' worldview.
One of the controversies associated with using test doubles is that they hardcode system behaviour into the test case. Because of this potential disadvantage, the ``classical'' view prefers to avoid test doubles when possible and instead use actual objects, though they are pragmatic enough to use doubles when required to. Martin~\cite{martin14:_when_mock}, for instance, puts forward the classical view in his blog entry ``When to Mock'', and Fowler~\cite{fowler07:_mocks_arent_stubs} states that he leans more towards the classical view himself, though in his article, he aims to provide what is, to him, the best possible argument for a mockist view. The ``mockist'' view, as advocated by the inventors of mock objects~\cite{freeman04:_mock_roles_objec} among others, strongly prefers test doubles; they will prefer creating a mock even if the real object is available and easy to create. They point out that mock implementations, being partial, can be more lightweight than stub fake implementations.

\todo[inline]{give another example of a case where we can use a real object instead of a test double}

Our work aims, in part, to determine how easy it is to avoid test doubles. The implication, if it is often easy to avoid doubles, is that developers can freely choose either the ``classical'' view or the ``mockist'' view, and if they take the ``classical'' view, then they do not often have to write mocks. On the other hand, if it is often difficult to avoid test doubles, then the ``mockist'' view may be a more comfortable worldview, because the developer will have to write a lot of mocks regardless. Of course, a particular system's design may be more amenable to a particular view.

\todo[inline]{hard to mock}


\paragraph{Respecting boundaries} A secondary goal of our work is to determine mock objects' distance, in the following sense.
The mockist point of view, as expressed
in~\cite{freeman04:_mock_roles_objec}, is ``only mock types you own'' and ``only mock your immediate
neighbours'', i.e. the opposite of ``mock objects should only be used
at the boundaries of the system''. In the example above, we saw that the
\texttt{JunitCreatorModel} test was using a mock for its immediate collaborator
workspace object. Mockists recommend using thin wrappers (stubs or fakes) for 
library objects. On the other hand, the classical view acknowledges that test doubles
are necessary for collaborators like databases and email-sending-code, which definitely
do not share ownership with the class being tested.

\todo[inline]{Can give an example of distance here}

%% also complicates the static analysis of test case source code. 
%% Such static analyses can help IDEs provide better
%% support to test case writers; enable better static estimation of test coverage
%% (avoiding mocks); and detect focal methods in test cases. While researchers have
%% proposed techniques for automatically generating mocks~\cite{alshahwan10:_autom,fazzini20:_framew_autom_test_mockin_mobil_apps}, our goal here is the opposite:
%% we detect mocks that already exist in test cases.

% Users of model checkers like CBMC often replace functions lower in
% the call graphs with mocks in order to allow the tool to reason
% about functions higher in the call graph.  This is dangerous because
% it isn't possible to then check that the mock truly is an
% abstraction of the code is stands for. That is part of the
% motivation for writing function contracts. If we can infer those
% contracts from the mocks that would be very valuable.

