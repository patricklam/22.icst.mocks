\documentclass[conference]{IEEEtran}
\IEEEoverridecommandlockouts
% The preceding line is only needed to identify funding in the first footnote. If that is unneeded, please comment it out.
\usepackage{url}
\usepackage{hyperref}
\usepackage{cite}
\usepackage{amsmath,amssymb,amsfonts}
\usepackage{algorithmic}
\usepackage{graphicx}
\usepackage{textcomp}
\usepackage{xcolor}
\usepackage{todo}
\usepackage{tikz}
\usepackage{subfig}
\def\BibTeX{{\rm B\kern-.05em{\sc i\kern-.025em b}\kern-.08em
    T\kern-.1667em\lower.7ex\hbox{E}\kern-.125emX}}

\tikzstyle{block} = [rectangle, draw, 
    text width=5em, text centered, rounded corners, minimum height=2em]
\tikzstyle{oval} = [ellipse, draw, 
    text width=5em, text centered, rounded corners, minimum height=2em]
\tikzstyle{bt} = [rectangle, draw, 
    text width=1em, text centered, rounded corners, minimum height=2em]
\usetikzlibrary{calc}
\usetikzlibrary{arrows.meta}
\usetikzlibrary{positioning}
\usetikzlibrary{fit}
\usetikzlibrary{shapes.geometric}

\begin{document}

\title{Do you have to mock me? An empirical study of mock usage in Java test suites}

%% \author{\IEEEauthorblockN{Mohammad Mahdi Abdollahpour, Qian Liang, and Patrick Lam\IEEEauthorrefmark{1}}
%% \IEEEauthorblockA{
%% \begin{tabular}{cc}
%% \IEEEauthorrefmark{1}University of Waterloo\\
%% Waterloo, ON, Canada \\
%% {\{mm2abdol,q8liang,patrick.lam\}@uwaterloo.ca}}
%% \end{tabular}
%% }
%% \maketitle
\author{\IEEEauthorblockN{Author 1, Author 2, and Author 3\IEEEauthorrefmark{1}}
\IEEEauthorblockA{
\begin{tabular}{cc}
\IEEEauthorrefmark{1}Institution 1\\
{\{author1, author2, author3\}@anon.org}}
\end{tabular}
}
\maketitle

\begin{abstract}
Mocks are commonly used in Java test suites, but are not without controversy. In this work we empirically investigate the prevalence of mock objects in Java test suites and estimate the proportion of tests that use mock libraries to create mock objects in a way that is not easily replaceable by stub objects.
\end{abstract}

\begin{IEEEkeywords}
static program analysis,
empirical studies,
mock objects,
software maintenance,
unit tests,
test suite design
\end{IEEEkeywords}

\section{Introduction}
Our goal is twofold: 1) we estimate the prevalence of mocks in extant test suites; and 2) we estimate how often it would be difficult to use alternatives to mocks (e.g. stubs).

percentage of classes being mocked (classes as mock subjects; how many distinct implementations of classes; how many tests have mocks)

is it the case that tests in package X are more likely to use real objects from X and mock objects from Y? Does it make sense to compute a notion of distance between X and Y? (e.g. you mock a database which is “far away” but not something close by)
how much dependence from main code and how much dependence from test case?

what is the lifetime of a mock object? does it survive across tests (in a field), or is it created just within a test? (compare to non-mocks)

What things are hard to mock?

https://blog.cleancoder.com/uncle-bob/2014/05/10/WhenToMock.html




\section{Design Considerations for Using Mocks}
We next provide some background information so that we can better describe ``classical'' and ``mockist'' views.
Meszaros~\cite{meszaros2007xunit} (and, following him, Fowler~\cite{fowler07:_mocks_arent_stubs}) use the term ``test double'' for the concept of a collaborator object required during testing. When testing a given class \texttt{C}, a developer will need to somehow provide instances of \texttt{C}'s collaborators to exercise \texttt{C}'s functionality. 
One of our key goals is to understand how free developers are to choose between different kinds of test doubles---notably, mocks and stubs/fakes. 

Returning to the original sources, in the first publication on mock objects~\cite{mackinnon00:_endo_testin}, the motivating example is a test case exercising a \texttt{JunitCreatorModel} object; that object takes a workspace object as a parameter to its constructor.

\begin{lstlisting}
  // ...
  JUnitCreatorModel creatorModel = new
        JunitCreatorModel(myMockWorkspace, PACKAGE_NAME);
  try {
    creatorModel.createTestCase
        (EXISTING_CLASS_NAME);
    // ...
\end{lstlisting}

This test needs a workspace object to exercise the model. (In this case, it is not possible to pass \texttt{null} as the constructor parameter, because the \texttt{JunitCreatorModel} calls methods on the workspace.) Stubs, fakes, and mocks are some of the kinds of test doubles. Following Meszaros, a stub's methods return canned answers, while a fake is a simplified but working implementation (e.g. an in-memory database rather than one that stores data to persistent storage). 

(Note that the definition of mock objects has shifted since the initial publication~\cite{mackinnon00:_endo_testin}; their use of the term ``mock'' includes what Meszaros would later call stubs and fakes. In their subsequent work~\cite{freeman04:_mock_roles_objec}, they still refer to stubs, but discourage explicit stub implementations and encourage the recording of mock object behaviours, as seen in their early mock framework jMock as well as modern mock frameworks including Mockito, PowerMock, and EasyMock.)

With the introduction of the concept of mocks came two worldviews mentioned above: the ``classical'' worldview and the ``mockist'' worldview.
One of the controversies associated with using test doubles is that they hardcode system behaviour into the test case. Because of this potential disadvantage, the ``classical'' view prefers to avoid test doubles when possible and instead use actual objects, though they are pragmatic enough to use doubles when required to. Martin (also known as ``Uncle Bob'')~\cite{martin14:_when_mock}, for instance, puts forward the classical view in his blog entry ``When to Mock'', and Fowler~\cite{fowler07:_mocks_arent_stubs} states that he leans more towards the classical view himself, though in his article, he aims to provide what is, to him, the best possible argument for a mockist view. The ``mockist'' view, as advocated by the inventors of mock objects~\cite{freeman04:_mock_roles_objec} among others, strongly prefers test doubles; they will prefer creating a mock---to elucidate the collaboration interface---even if the real type is easy to create. They point out that mock implementations, being partial, can be more lightweight than stub fake implementations.

As an example, the \texttt{IteratorUtilsTest} from commons-collection-4.4 provides a number of tests for its \texttt{IteratorUtils} class. Some of the tests work with an array of mock \texttt{org.w3c.dom.Node} objects. Such objects would be hard to create directly (the \texttt{Node} interface has 37 methods), and so if we wanted to have \texttt{Node} objects it would indeed be easiest to create mocks. However, in fact, the tests don't rely on any properties of \texttt{Node}s, so they could populate the iterators being tested with real \texttt{Object} objects instead.

Our work thus aims, in part, to determine how easy it is to avoid test doubles. The implication, if it is often easy to avoid doubles, is that developers can freely choose either the ``classical'' view or the ``mockist'' view, and if they take the ``classical'' view, then they do not often have to write mocks. On the other hand, if it is often difficult to avoid test doubles, then the ``mockist'' view may be a more comfortable worldview, because the developer will have to write a lot of mocks regardless. Of course, a particular system's design may be more amenable to a particular view.

%\todo[inline]{hard to mock}


\paragraph{Respecting boundaries} A secondary goal of our work is to determine mock objects' distance, in the following sense.
The mockist point of view, as expressed
in~\cite{freeman04:_mock_roles_objec}, is ``only mock types you own'' and ``only mock your immediate
neighbours'', i.e. the opposite of ``mock objects should only be used
at the boundaries of the system''. 

In the example above, we saw that the
\texttt{JunitCreatorModel} test was using a mock for its immediate collaborator
workspace object. Mockists recommend using thin wrappers (stubs or fakes) for 
library objects. On the other hand, the classical view uses test doubles
for collaborators like databases and email-sending-code, creating them using mock libraries, even if such collaborators definitely
do not share ownership with the class being tested.

%\todo[inline]{Can give an example of distance here}

%% also complicates the static analysis of test case source code. 
%% Such static analyses can help IDEs provide better
%% support to test case writers; enable better static estimation of test coverage
%% (avoiding mocks); and detect focal methods in test cases. While researchers have
%% proposed techniques for automatically generating mocks~\cite{alshahwan10:_autom,fazzini20:_framew_autom_test_mockin_mobil_apps}, our goal here is the opposite:
%% we detect mocks that already exist in test cases.

% Users of model checkers like CBMC often replace functions lower in
% the call graphs with mocks in order to allow the tool to reason
% about functions higher in the call graph.  This is dangerous because
% it isn't possible to then check that the mock truly is an
% abstraction of the code is stands for. That is part of the
% motivation for writing function contracts. If we can infer those
% contracts from the mocks that would be very valuable.



\section{Empirical Study: Methodology}
How we do things.


\section{Empirical Results}
Results go here.

\begin{table}[h]
    \begin{tabular}{lrrrrrr}
        Benchmark                   & \multirow{2}{*}{\shortstack{\# of Mocked \\ Types}} & C1 & C2 & C3 & C4 & C5 \\ 
         & & & & & & \\ \hline
        RxRelay                     & 2                 & 0  & 0  & 0  & 0  & 0  \\
        braintree\_java             & 6                 & 0  & 0  & 0  & 0  & 0  \\
        frontend-maven-plugin       & 2                 & 0  & 0  & 0  & 0  & 0  \\
        spark                       & 10                & 0  & 4  & 2  & 0  & 0  \\
        lambda                      & 11                & 0  & 1  & 7  & 0  & 0  \\
        thumbnailator               & 14                & 0  & 0  & 2  & 0  & 2  \\
        azure-functions-java-worker & 3                 & 3  & 0  & 3  & 0  & 0  \\
        jdeb                        & 2                 & 0  & 0  & 0  & 0  & 0  \\
        maven-dependency-plugin     & 6                 & 0  & 3  & 1  & 0  & 1  \\
        openshift-restclient-java   & 16                & 0  & 12 & 0  & 0  & 0  \\
        javapoet                    & 3                 & 0  & 2  & 0  & 0  & 0  \\
        exec-maven-plugin           & 3                 & 0  & 1  & 0  & 0  & 1  \\
        plexus-resources            & 2                 & 0  & 1  & 0  & 0  & 0  \\
        maven-assembly-plugin       & 18                & 0  & 8  & 0  & 0  & 1  \\
        java-faker                  & 7                 & 0  & 0  & 2  & 0  & 0  \\
        minimal-json                & 6                 & 0  & 0  & 3  & 0  & 0  \\
        JSON-java                   & 3                 & 0  & 1  & 0  & 0  & 0  \\
        java-gitlab-api             & 2                 & 0  & 0  & 1  & 0  & 0  \\
        lettuce-core                & 43                & 0  & 14 & 3  & 1  & 1  \\
        zip4j                       & 11                & 0  & 3  & 2  & 0  & 0  \\
        versions-maven-plugin       & 14                & 0  & 8  & 1  & 0  & 1  \\
        mbassador                   & 3                 & 0  & 0  & 0  & 0  & 0
    \end{tabular}
\end{table}
% \begin{table}[]
%     \begin{tabular}{lrr}
%         Benchmark                   & tests & tests using mocks \\
%         RxRelay                     & 64    & 13                \\
%         braintree\_java             &       & 9                 \\
%         frontend-maven-plugin       & 30    & 2                 \\
%         spark                       & 271   & 94                \\
%         lambda                      &       & 45                \\
%         thumbnailator               & 707   & 57                \\
%         azure-functions-java-worker & 3     & 2                 \\
%         jdeb                        & 10    & 4                 \\
%         maven-dependency-plugin     & 33    & 18                \\
%         openshift-restclient-java   & 16    & 42                \\
%         javapoet                    &       & 4                 \\
%         exec-maven-plugin           & 19    & 1                 \\
%         plexus-resources            & 3     & 2                 \\
%         maven-assembly-plugin       &       & 100               \\
%         java-faker                  & 524   & 503               \\
%         minimal-json                & 462   & 21                \\
%         JSON-java                   & 373   & 2                 \\
%         java-gitlab-api             & 14    & 6                 \\
%         lettuce-core                &       & 206               \\
%         zip4j                       & 325   & 37                \\
%         versions-maven-plugin       & 90    & 16                \\
%         mbassador                   & 53    & 8
%     \end{tabular}
% \end{table}

\begin{table*}[]
    \newcolumntype{L}{>{\raggedright\arraybackslash}X}
    \begin{tabular*}{\linewidth}[t]{lrrrrrr}
        \multirow{2}{*}{Benchmark}                   & \multirow{2}{*}{\shortstack{\# of Mocked Types \\ (Total)}} & \multirow{2}{*}{\shortstack{\# of Interfaces \\ (Internal/External)}} & \multirow{2}{*}{\shortstack{\# of Abstract Classes \\ (Internal/External)}} & \multirow{2}{*}{\shortstack{\# of Concrete Classes \\ (Internal/External)}} & \multirow{2}{*}{\shortstack{\# of Tests \\ (Total)}} & \multirow{2}{*}{\shortstack{\# of Tests \\ (using Mocks)}} \\
        & & & & & & \\ \hline
        RxRelay                     & 2                  & 0 / 1                 & 0 / 1                & 0 / 0    & 64    & 13                   \\ 
        braintree\_java             & 6                  & 1 / 0                 & 3 / 1                & 1 / 0    &       & 9                    \\ 
        frontend-maven-plugin       & 2                  & 0 / 0                 & 0 / 2                & 0 / 0    & 30    & 2                    \\ 
        spark                       & 10                 & 2 / 3                 & 4 / 1                & 0 / 0    & 271   & 94                   \\ 
        lambda                      & 11                 & 1 / 2                 & 7 / 1                & 0 / 0    &       & 45                   \\ 
        thumbnailator               & 14                 & 5 / 0                 & 2 / 2                & 1 / 4    & 707   & 57                   \\ 
        azure-functions-java-worker & 3                  & 0 / 0                 & 0 / 3                & 0 / 0    & 3     & 2                    \\ 
        jdeb                        & 2                  & 1 / 0                 & 0 / 1                & 0 / 0    & 10    & 4                    \\ 
        maven-dependency-plugin     & 6                  & 0 / 3                 & 0 / 3                & 0 / 0    & 33    & 18                   \\ 
        openshift-restclient-java   & 16                 & 16 / 0                & 0 / 0                & 0 / 0    & 16    & 42                   \\ 
        javapoet                    & 3                  & 0 / 2                 & 0 / 1                & 0 / 0    &       & 4                    \\ 
        exec-maven-plugin           & 3                  & 0 / 1                 & 0 / 2                & 0 / 0    & 19    & 1                    \\ 
        plexus-resources            & 2                  & 0 / 1                 & 0 / 1                & 0 / 0    & 3     & 2                    \\ 
        maven-assembly-plugin       & 18                 & 4 / 11                & 0 / 3                & 0 / 0    &       & 100                  \\ 
        java-faker                  & 7                  & 0 / 0                 & 6 / 1                & 0 / 0    & 524   & 503                  \\ 
        minimal-json                & 6                  & 0 / 0                 & 2 / 1                & 1 / 2    & 462   & 21                   \\ 
        JSON-java                   & 3                  & 2 / 0                 & 0 / 1                & 0 / 0    & 373   & 2                    \\ 
        java-gitlab-api             & 2                  & 0 / 0                 & 0 / 2                & 0 / 0    & 14    & 6                    \\ 
        lettuce-core                & 43                 & 1 / 30                & 0 / 9                & 0 / 3    &       & 206                  \\ 
        zip4j                       & 11                 & 0 / 7                 & 0 / 2                & 0 / 2    & 325   & 37                   \\ 
        versions-maven-plugin       & 14                 & 0 / 10                & 0 / 3                & 0 / 1    & 90    & 16                   \\ 
        mbassador                   & 3                  & 1 / 0                 & 1 / 1                & 0 / 0  & 53    & 8
    \end{tabular*}
\end{table*}


% \begin{table}[]
%     \begin{tabular}{lrrrrrr}
%         Benchmark                   & mocked types & c1 & c2 & c3 & c4 & c5 \\
%         RxRelay                     & 2            & 0  & 0  & 0  & 0  & 0  \\
%         braintree\_java             & 6            & 0  & 0  & 0  & 0  & 0  \\
%         frontend-maven-plugin       & 2            & 0  & 0  & 0  & 0  & 0  \\
%         spark                       & 10           & 0  & 4  & 2  & 0  & 0  \\
%         lambda                      & 11           & 0  & 1  & 7  & 0  & 0  \\
%         thumbnailator               & 14           & 0  & 0  & 2  & 0  & 2  \\
%         azure-functions-java-worker & 3            & 3  & 0  & 3  & 0  & 0  \\
%         jdeb                        & 2            & 0  & 0  & 0  & 0  & 0  \\
%         maven-dependency-plugin     & 6            & 0  & 3  & 1  & 0  & 1  \\
%         openshift-restclient-java   & 16           & 0  & 12 & 0  & 0  & 0  \\
%         javapoet                    & 3            & 0  & 2  & 0  & 0  & 0  \\
%         exec-maven-plugin           & 3            & 0  & 1  & 0  & 0  & 1  \\
%         plexus-resources            & 2            & 0  & 1  & 0  & 0  & 0  \\
%         maven-assembly-plugin       & 18           & 0  & 8  & 0  & 0  & 1  \\
%         java-faker                  & 7            & 0  & 0  & 2  & 0  & 0  \\
%         minimal-json                & 6            & 0  & 0  & 3  & 0  & 0  \\
%         JSON-java                   & 3            & 0  & 1  & 0  & 0  & 0  \\
%         java-gitlab-api             & 2            & 0  & 0  & 1  & 0  & 0  \\
%         lettuce-core                & 43           & 0  & 14 & 3  & 1  & 1  \\
%         zip4j                       & 11           & 0  & 3  & 2  & 0  & 0  \\
%         versions-maven-plugin       & 14           & 0  & 8  & 1  & 0  & 1  \\
%         mbassador                   & 3            & 0  & 0  & 0  & 0  & 0
%     \end{tabular}
% \end{table}

\section{Related Work}
\label{sec:related}

Empirical use of language features papers. Rust unsafe.

We discuss related work in the areas of focal method detection,
% declarative versus imperative static analysis,
treatment of containers, and taint analysis.

\paragraph{Focal methods and classes} To situate Ghafari et al's work~\cite{ghafari15:_autom}, a number of previous works have studied test-to-code traceability by identifying focal \emph{classes} for a test case---the classes which are tested by a test case. The focal \emph{methods} we discuss in this paper belong to focal classes. Ghafari et al were the first to extend the study of traceability to focal methods. Before that, Qusef et al proposed techniques which identify focal classes. In~\cite{DBLP:conf/icsm/QusefBOLB11}, they propose a two-stage approach relying on the assumption that the last assertion statement in a test case is the key assertion. The first stage uses dynamic slicing to find all classes that contribute to the values tested in the assertion (possibly including mocks), while the second stage filters classes and keeps only those textually closest to the test class. An additional mock object filter would help remove definitely-not-focal classes. Earlier work by Qusef et al~\cite{DBLP:conf/icsm/QusefOL10} uses dataflow analysis instead of dynamic slicing.

Rompaey and Demeyer~\cite{rompaey09:_estab_traceab_links_unit_test} evaluate six other heuristics for finding focal classes: naming conventions, types referred to in tests (``fixture element types''), the static call graph, the last call before the assert, lexical analysis of the code and the test, and co-evolution of the test and the main code. No heuristic dominates: different heuristics work for different codebases. Our approach adds another way to rule out unwanted focal method and focal class results.

Ying and Tarr~\cite{DBLP:conf/eclipse/YingT07} also propose heuristics to filter out unwanted methods during code inspection. Their heuristics are based on characteristics of the call graph, i.e. they filter out small methods and methods closer to the bottom of a call graph, depending on tuneable parameters. These heuristics empirically eliminate mock calls in their benchmarks, but there is no principled reason for that to be the case, and indeed, the static call graph that they depend on should not interact well with mock calls.

\paragraph{Treatment of containers} In this work, we use coarse-grained abstractions for containers, consistent with the approach from Chu et al~\cite{chu12:_collec_disjoin_analy}. We do not observe sophisticated container manipulations where it would be necessary to track exactly which elements of a container are mocks. Were such an analysis necessary, the fine-grained container client analysis by Dillig et al~\cite{dillig11:_precis_reason_progr_using_contain} would work.

\paragraph{Taint analysis} Like many other static analyses, our mock analysis can be seen as a variant of a static taint analysis: sources are mock creation methods, while sinks are method invocations. There are no sanitizers in our case. However, for a taint analysis, there is usually a small set of sink methods, while in our case, every method invocation in a test method is a potential sink. In some ways, our analysis resembles an information flow analysis like that by Clark et al~\cite{clark07:_static_analy_quant_infor_flow}. However, the goal of our analysis (detecting possible mocks) is different from taint and information flow analyses in that it is not security-sensitive, so the balance between false positives and false negatives is different---it is less critical to not miss any potential important mock invocations, whereas missing a whole class of tainted methods would often be unacceptable.


%% \paragraph{Imperative vs declarative}
%% Kildall contributed perhaps the first dataflow analysis~\cite{kildall73:_unified_approac_global_progr_optim} as the concept is understood today, describing an algorithm for intraprocedural constant propagation and common subexpression elimination. His algorithm, operating on the program graph, is described in quite imperative pseudocode (and proven to terminate). In some sense, implementing algorithms imperatively is the default, and doesn't need further discussion, except to point out that program analysis frameworks such as Soot~\cite{Vallee-Rai:1999:SJB:781995.782008} provide libraries that can ease the implementation burden.

%% To our knowledge, Corsini et al did some of the first work in declarative program analysis~\cite{corsini93:_effic}; however, that work performed abstract interpretation on (tiny) logic programs rather than imperative programs. Dawson et al~\cite{dawson96:_pract_progr_analy_using_gener} did similar work. Around the same time, Reps proposed~\cite{Reps1995} a declarative analysis to perform demand versions of interprocedural program analyses, which is similar to what we have here; however, we compute all of the analysis results rather than performing a demand analysis. CodeQuest by Hajijev et al~\cite{hajiyev06} also allows developers to perform AST-level code queries using a declarative query language. {\sc Dimple$^+$}~\cite{benton07:_inter_scalab_declar_progr_analy}\cite[Chapter 3]{benton08:_fast_effec_progr_analy_objec_level_paral} by Benton and Fischer may be closest to what we are advocating as the declarative analysis approach. While Benton's dissertation presents a simple {\sc Dimple$^+$} implementation of Andersen's points-to analysis, the {\sc Dimple$^+$} work does not have Doop's sophisticated pointer analysis available to it. Soufflé, by Scholz et al~\cite{scholz16:_fast_large_scale_progr_analy_datal}, advocates for declarative static analysis (but without comparing it directly to an imperative approach as we do here), and presents performance optimizations needed to achieve this goal.
%% Finally, Doop~\cite{bravenboer09:_stric_declar_specif_sophis_point_analy}, which is now primarily implemented with a Soufflé backend, is perhaps the most powerful extant declarative program analysis, and focusses on expressing sophisticated pointer analyses in Datalog. 



%% % implementation note: Reps's approach is much more complicated than what we have in Doop. Perhaps Doop's use of SSA and simulation of phi nodes allows it to use much simpler rules, or maybe it's the specific analyses that are being implemented. e.g. for Doop, which computes an overapproximation, merging the two branches using the virtual phi node (simulated as "x = phi(x1,x2) => x = x1; x = x2") works just fine.

%% In terms of comparing implementations, Prakash et al~\cite{prakash21:_effec_progr_repres_point_analy} compare pointer analysis as provided by Doop and Wala; in some sense, the present work is similar to that work in that both works compare two frameworks. However, that work compares empirical results from two families of pointer analysis implementations (and finds that the specific intermediate representation used doesn't change the results much), while we discuss the process of implementing a static analysis declaratively versus imperatively. Like us, they note that Doop is difficult to incorporate into a program transformation framework (it works better in standalone mode) while Wala's results are readily available; a similar result applies to any result that a Soot-based data flow analysis produces as compared to a Doop-based declarative analysis.




\section{Discussion}
Some discussion here.


Threat to validity: Java versus C\#.

\todo[inline]{yet in C\# mocking is done completely different. Moq is the popular framework and it does mocking via inheritance (not bytecode manipulation). It can only mock interfaces and class methods that are virtual (can be overridden by a subclass). 
so back to your statement.. if in C\# mocking is done mostly for things that can be easily stubbed (unless they have too many methods) why is mocking popular there?}


\bibliographystyle{plainurl}
\bibliography{bibliography}

\end{document}
