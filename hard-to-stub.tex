RQ2 asks about, for the objects that are mocked, whether they would be easy to stub. 
We define some predicates which approximate ``easy to stub'', and explain them in this stub.

%% One challenge for whole-program static analysis of any test-related code is that whole-program
%% analyses tend to aggressively prune code that is never invoked. Tests are invoked via reflection
%% by a test driver. Hence, to analyze tests, we ran a first analysis to create a static test driver.

%\paragraph{Counting classes that are hard to stub}

% We used Doop queries to count classes that fit our definition of being hard to stub. 
We developed a data analysis codebase on top of a selection of Doop facts (\texttt{Var-Type}, \texttt{ClassModifier}, \texttt{Method}, \texttt{Method-Modifier}, \texttt{Param-Annotation}) and two of our mock analysis outputs (\texttt{isMockVar}, \texttt{isMockInvocation}). At its core, this codebase includes queries to filter out types that fit our definition of being hard to stub, in addition to the queries for computing aggregate statistics for each benchmark.

We explain why we think these predicates express facets of being difficult to stub.



%% When is it difficult not to use mocks?

%% [Question: Difficult to Stub =? Difficult not to Mock]

%% (N = <int>, N_A = <int>, AC = Abstract Class, C = Class, I = Interface, DtS = Difficult to Stub, ctor = Constructor, *internal)
%% For any type T, if one (or more) of the below conditions are satisfied, then we classify T as DtS:
%% as C:  T is final
%% as I: T has more than N methods
%% as C or AC: T has no public ctor
%% as C or AC: T has final method M which is (needed to be) invoked in at least one test method.
%% as C or AC: T has only “annoying” public ctor(s). 
%% public ctor PC is “annoying” if one or more of the following conditions are satisfied:
%% PC has more than N_A arguments
%% PC has an argument A and A is “not nullable” and A’s type is DtS
%% argument A is not nullable if A’s type is primitive or A is annotated by @NotNull

%% *as C or AC or I: T has method (or public ctor) M and M is (needed to be) invoked in at least one test case and it is computationally expensive to execute any meaningful implementation of M.
%% [+special case about mocking Object and primitive types (+their collections) -> convenience if not condition 6]


%% alternative for 5.b: there will be errors if we modify the program and add @Nullable annotation to A
%% Checker framework will complain about any dereference of possibly-null reference types by default. (Adding @Nullable will only add checks for any nonnull to nullable assignments (source).) 
%% So if there is a possibly-null dereference error on a constructor argument C_A (or any variable pointing to C_A), then we can not pass null to easily initialize it.
%% parsing checker outputs takes a bit of work! (is there a way to export maven output into a better format like csv or json?)

%% about 4: mockito did not support stubbing final methods before version 2. after version 2 devs should activate a plugin to be able to stub final methods (source1, source2). 
%% there is no case of stubbing a final method in our benchmarks.


%% currently we have (or can super easily get) count/percentage for:
%% 1, 2 (if we decide on N), 3, 4 (have the code but no benchmark), 5.a
