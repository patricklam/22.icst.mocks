\label{sec:motivating-example}

Our core \textsc{MockDetector} tool finds variables containing mock objects, along with invocations on mock objects, in unit tests. It does so by identifying invocations on variables which have been assigned an object flowing from a mock creation site.

\begin{lstlisting}[basicstyle=\ttfamily, caption={Code snippet from maven-core, where calls to both the focal method \texttt{getToolchainsForType()} and to mock \texttt{session}'s \texttt{getRequest()} method occur in test \textit{testMisconfiguredToolchain()}.},
numbers=left,numbersep=2pt,basicstyle=\scriptsize\ttfamily,language = Java, framesep=4.5mm, escapechar=|,
framexleftmargin=1.0mm, captionpos=b, label=lis:mockCall, morekeywords={@Test}]
@Test public void testMisconfiguredToolchain() throws Exception {
  MavenSession session = mock( MavenSession.class );
  MavenExecutionRequest req = new DefaultMavenExecutionRequest();
  when( session.getRequest() ).thenReturn( req ); |\label{line:mock}|

  ToolchainPrivate[] basics =
    toolchainManager.getToolchainsForType("basic", session); |\label{line:real}|

  assertEquals( 0, basics.length );
}
\end{lstlisting}

To motivate our approach, consider Listing~\ref{lis:mockCall}, which presents a unit test from the Maven project. We can observe that this test creates a mock \texttt{MavenSession} but a real \texttt{DefaultMavenExecutionRequest}, suggesting that it follows the classical view more than the mockist view (though these aren't binaries)---it pragmatically mocks some but not all of the dependencies. Class \texttt{MavenExecutionRequest} could be a mock, but there is no real advantage to doing so, even under a pure mockist point of view, because it is easy to create and does not directly collaborate with the \texttt{ToolchainManager} class under test. The focal method~\cite{ghafari15:_autom} being tested here is \texttt{getToolchainsForType()}, called on Line~\ref{line:real}; the mock \texttt{session} object has a recorded expectation (Line~\ref{line:mock}) that its \texttt{getRequest()} method be invoked. The \texttt{MavenSession} class (not shown) does not satisfy our difficult-to-stub criteria since it is a non-final class with invokable constructors; however, calling one of those constructors requires at least 4 other objects, so it is not trivial to stub.

We discuss the contents of this test method. Line~\ref{line:mock} calls \texttt{getRequest()}, invoking it on mock object \texttt{session}. At the bytecode level, that call to the mock's \texttt{getRequest()} is indistinguishable to a real call with respect to mockness; the static call graph contains a call to the actual method in both cases, so that current static analysis tools cannot easily tell the difference between the method invocation on a mock object on line~\ref{line:mock} and the method invocation on a real object on line~\ref{line:real}. This uncertainty about mockness confounds a naive static analysis that attempts to identify focal methods. For instance, Ghafari et al~\cite{ghafari15:_autom}'s heuristic would fail on this test, as it returns the last mutator method in the object under test, and the focal method here is an accessor. Mockness information can also help IDEs provide better suggestions. 
%Section~\ref{sec:focal} discusses applications of mock analysis to finding focal methods in more detail. 

\begin{figure}[ht]
        \begin{lstlisting}[basicstyle=\ttfamily,
        numbers=left,numbersep=0pt,basicstyle=\scriptsize\ttfamily,language = Java, framesep=4.5mm, framexleftmargin=1.0mm, captionpos=b, escapechar=|, morekeywords={@Test}]
        //      mock: |\xmark~\,|  mockAPI: |\xmark|
        Object object1 = new Object();
        
        // mock: |\xmark|
        object1.foo();
        
        //      mock: |\cmark|   mockAPI: |\cmark|
        Object object2 = mock(Object.class);
        
        // mock: |\cmark|
        object2.foo();
        \end{lstlisting}
        %    \includegraphics[width=.25\textwidth]{Images/mockInvocationIllustration.png}       
        \caption{Our static analysis propagates mockness from sources (e.g. \texttt{mock(Object.class}) to invocations.}
        \label{fig:mockMethodIllustration}
        
\end{figure}

\begin{lstlisting}[basicstyle=\ttfamily, caption={Jimple Intermediate Representation for some code in Figure~\ref{fig:mockMethodIllustration}
.},
numbers=left,numbersep=2pt,basicstyle=\scriptsize\ttfamily, captionpos=b, label=lis:mockMethodIllustrationIR, escapechar=|, morekeywords={@T
est, specialinvoke, virtualinvoke, staticinvoke}]
java.lang.Object $r1, r2;

$r1 = new java.lang.Object; |\label{line:lis3line3}|
specialinvoke $r1.<java.lang.Object: void <init>()>(); |\label{line:lis3line4}|
virtualinvoke $r1.<java.lang.Object: void foo()>(); |\label{line:lis3line5}|
r2 = staticinvoke <org.mockito.Mockito: java.lang.Object
        mock(java.lang.Class)> (class "Ljava/lang/Object;"); |\label{line:lis3line6}|
virtualinvoke r2.<java.lang.Object: void foo()>(); |\label{line:lis3line9}|
\end{lstlisting}


\paragraph{Basic declarative analysis} Figure~\ref{fig:mockMethodIllustration} provides a simplified example of a mock-using test. Listing~\ref{lis:mockMethodIllustrationIR} provides the Jimple Intermediate Representation, created by Soot, for that example. We ask whether the call on IR line~\ref{line:lis3line9} satisfies the predicate \texttt{isMockInvocation} (facts Listing~\ref{lis:facts}, line~\ref{line:facts-imi}), which we define to hold the analysis result (all mock invocation sites in the program). It does, because of facts lines~\ref{line:facts-vmi}--\ref{line:facts-imv}: IR line~\ref{line:lis3line9} contains a virtual method invocation, and receiver \texttt{r2} for the invocation on that line satisfies our predicate \texttt{isMockVar}, which holds all mock-containing variables in the program (Section~\ref{sec:methodology} provides more details). Predicate \texttt{isMockVar} holds because of lines~\ref{line:facts-arv}--\ref{line:facts-cms}: \texttt{r2} satisfies \texttt{isMockVar} because IR line~\ref{line:lis3line6} assigns \texttt{r2} the return value from mock source method \texttt{createMock} (facts line~\ref{line:facts-arv}), and the call to \texttt{createMock} satisfies predicate \texttt{callsMockSource} (facts line~\ref{line:facts-cms}), which requires that the call destination \texttt{createMock} be enumerated as a constant in our 1-ary relation \texttt{MockSourceMethod} (facts line~\ref{line:facts-msm}), and that there be a call graph edge between the method invocation at line~\ref{line:lis3line6} and the mock source method (facts line~\ref{line:facts-cge}).


\begin{lstlisting}[basicstyle=\ttfamily, caption={Facts about invocation \texttt{r2.foo()} in method \texttt{test}.},
numbers=left,numbersep=2pt,basicstyle=\scriptsize\ttfamily, framesep=4.5mm, framexleftmargin=1.0mm, captionpos=b, label=lis:facts, escapechar=!, morekeywords={@Test}]
isMockInvocation(<Object:void foo()>/test/0, 
                 <Object:void foo()>, test, _, r2). !\label{line:facts-imi}!
|VirtualMethodInvocation(<Object:void foo()>/test/0, !\label{line:facts-vmi}!
|                        <Object:void foo()>, test).
|VirtualMethodInvocation_Base(<Object:void foo()>/test/0, r2).
|isMockVar(r2). !\label{line:facts-imv}!
|-AssignReturnValue(<Mockito:Object mock(Class)>/test/0!\label{line:facts-arv}!, r2).
|-callsMockSource(<Mockito:Object mock(Class)>/test/0). !\label{line:facts-cms}!
|MockSourceMethod(<Mockito:Object mock(Class)>). !\label{line:facts-msm}!
|CallGraphEdge(_,<Mockito:Object mock(Class)>/test/0, _,!\label{line:facts-cge}!
|              <Mockito: Object mock(Class)>). 
\end{lstlisting}


\paragraph{Extensions: arrays and collections} While designing \textsc{MockDetector}, we observed that developers store mock objects in arrays and collections. Listing~\ref{lis:container} presents method \textit{setUp()} in class \texttt{NodeListIteratorTest} from commons-collections-4.4. Line \ref{line:storeMocksInArray} puts mock \texttt{Node} objects in array field \texttt{nodes}, later used in tests. We have declarative constraints propagating mockness for assignment statements containing array reads or writes; if the value (local variables or field reference sources) on the opposite side of the assignment (the statement's destination or source) is a mock, then it examines the context. If any of these locals or fields involved are mocks, then the tool would mark the local or field reference as an array mock---it propagates mockness to the array container.

\begin{lstlisting}[basicstyle=\ttfamily, caption={This example illustrates a field array container holding mock objects from \textit{setUp()} in \texttt{NodeListIteratorTest}.},
numbers=left,numbersep=2pt,basicstyle=\scriptsize\ttfamily,language = Java, framesep=4.5mm, framexleftmargin=1.0mm, captionpos=b, label=lis:container, escapechar=|, morekeywords={@Test}]
// Node array to be filled with mock Node instances
private Node[] nodes;
@Test protected void setUp() throws Exception {
  // create mock Node Instances and fill Node[] to be used by tests
  final Node node1 = createMock(Element.class);
  final Node node2 = createMock(Element.class);
  final Node node3 = createMock(Text.class);
  final Node node4 = createMock(Element.class);
  nodes = new Node[] {node1, node2, node3, node4}; |\label{line:storeMocksInArray}|
  // ...
}
\end{lstlisting}

Figure~\ref{fig:arrayMockIllustration} illustrates a mock-containing array, and Listing~\ref{lis:arrayIllustrationIR} shows the corresponding IR. Our analysis reaches the mock API call on lines~\ref{line:lis4line4}--\ref{line:lis4line6}, where it records that \texttt{\$r2} is a mock object---it creates a MockStatus with mock bit set for \texttt{\$r2}. The tool would then handle the cast expression assigning to \texttt{r1} on line~\ref{line:lis4line7}, giving it the same MockStatus as \texttt{\$r2}. When the analysis reaches line~\ref{line:lis4line9}, it finds an array reference on the left hand side, along with \texttt{r1} stored in the array on the right hand side of the assignment statement. At that point, it has a MockStatus associated with \texttt{r1}, with the mock bit turned on. It can now deduce that \texttt{\$r3} on the LHS is an array container which may hold a mock. Therefore, \textsc{MockDetector} marks \texttt{\$r3} as a mock-containing array. Collections work similarly---we hard-code methods from the Java Collections API.

\begin{lstlisting}[basicstyle=\ttfamily, caption={Jimple Intermediate Representation for the array in Figure~\ref{fig:arrayMockIllustration}.},
numbers=left,numbersep=2pt,basicstyle=\scriptsize\ttfamily, framesep=4.5mm, framexleftmargin=1.0mm, captionpos=b, label=lis:arrayIllustrationIR, escapechar=|, morekeywords={@Test, specialinvoke, virtualinvoke, staticinvoke, newarray}]
java.lang.Object r1, $r2;
java.lang.Object[] $r3;

$r2 = staticinvoke <org.easymock.EasyMock: |\label{line:lis4line4}|
java.lang.Object createMock(java.lang.Class)>
(class "java.lang.Object;"); |\label{line:lis4line6}|
r1 = (java.lang.Object) $r2; |\label{line:lis4line7}|
$r3 = newarray (java.lang.Object)[1]; |\label{line:lis4line8}|
$r3[0] = r1;  |\label{line:lis4line9}|
\end{lstlisting}

\begin{figure}[h]
	\begin{lstlisting}[
	numbers=left,numbersep=0pt,basicstyle=\scriptsize\ttfamily,language = Java, framesep=4.5mm, framexleftmargin=1.0mm, captionpos=b, escapechar=|, morekeywords={@Test}]
	//       mock: |\cmark|    mockAPI: |\cmark|
	Object object1 = createMock(Object.class);
	
	// arrayMock: |\cmark| |$\Leftarrow$| array-write    |~~|  mock: |\cmark|
	objects |~| = new Object[]  |~|  { object1 };
	\end{lstlisting}
	
	\caption{Our static analysis also finds array mocks.}
	\label{fig:arrayMockIllustration}
	
\end{figure}

