A key challenge in writing tests for object-oriented systems is
finding collaborators for objects that are being tested. Sometimes,
collaborators are difficult to instantiate---for instance, because
they themselves have many dependencies---or, their execution might be
difficult to control, slow, or unreliable, as in the case of a
database connection.  Test doubles~\cite{meszaros2007xunit}, including
mock objects, stubs, and fakes, are often required even to instantiate
objects under test, as well as to test these objects' functionality.

Proper usage of mock objects and stubs/fakes has been debated among
practitioners at length, and there is a division between so-called
``classical'' and ``mockist'' schools. In the classical view, mocks
are just one useful tool among many; they enable developers to easily
provide partial implementations of collaborators. Generally, in the classical
view, mocks should only be created when they are indispensable, i.e. hard to avoid. In the mockist view,
which is quite Test-Driven Development (TDD)-centric, the primary value
of mocks is in guiding system design (towards, of course, more mock-friendly
systems). Concretely, this means that mockists advocate behaviour-based
tests that verify that an object under test makes the correct calls
to its collaborators. Many more collaborator objects will be mocked
(rather than stubbed or faked) under the mockist view.

Another aspect of the mockist view is that tests should use mocks
only for collaborators with the same (code) owner as the object
being tested. Under the classical view, developers will mock
when required, no matter if the collaborator to be mocked is in
the same module, or is instead in a library.

Our goal is to form a data-driven opinion about how extant designs of
Java systems support classical versus mockist views.  To do so, we
need to understand mock creation and usage in Java at a finer-grained level than in
previous work.

Specifically, we are interested in the following research questions:
\begin{itemize}
\item RQ1: Quantitatively, how are mocks used in practice? Which types do developers mock, and what is the
mock objects' relationships with the classes being tested?
\item RQ2: Are developers choosing to mock objects for which it would be otherwise easy to write stub objects, or
are they mainly writing mocks in cases where mocking is indispensable?
\end{itemize}
If developers write mocks for objects that are easy to stub, then we have some evidence that they are choosing
a mockist view. Conversely, if they are only writing mocks for objects that are difficult to stub, we could
expect them to be taking a classical view.

Our contributions include:
\begin{itemize}
\item a novel {\sc MockDetector} tool that performs a fine-grained static analysis to find mock object instantiations, determine which method calls
in test cases are mock invocations, and identify objects that are difficult to stub; and,
\item empirical data on a set of benchmarks about the use of mock objects in practice and the contexts in which these mocks are used.
\end{itemize}
We intend to make our tool and dataset publicly available.

%Our goal is twofold: 1) we estimate the prevalence of mocks in extant test suites; and 2) we estimate how often it would be difficult to use alternatives to mocks (e.g. stubs).

%RQ: statistics on mocks: percentage of classes being mocked (classes as mock subjects; how many distinct implementations of classes; how many tests have mocks)

%RQ: is it the case that tests in package X are more likely to use real objects from X and mock objects from Y? Does it make sense to compute a notion of distance between X and Y? (e.g. you mock a database which is “far away” but not something close by)
%how much dependence from main code and how much dependence from test case?

%RQ: what is the lifetime of a mock object? does it survive across tests (in a field), or is it created just within a test? (compare to non-mocks)

%RQ: What things are hard to mock?

%https://blog.cleancoder.com/uncle-bob/2014/05/10/WhenToMock.html


